\documentclass[letterpaper,11pt]{article}

\usepackage{latexsym}
\usepackage[empty]{fullpage}
\usepackage{titlesec}
\usepackage{marvosym}
\usepackage[usenames,dvipsnames]{color}
\usepackage{verbatim}
\usepackage{enumitem}
\usepackage[hidelinks]{hyperref}
\usepackage{fancyhdr}
\usepackage[english]{babel}
\usepackage{tabularx}
\usepackage{xcolor}
\usepackage{fontawesome5}
\usepackage{xcolor}
\usepackage{hyperref}

\hypersetup{
    colorlinks,
    linkcolor={blue},
    citecolor={blue},
    urlcolor={blue}
}

\input{glyphtounicode}

% -------------------- FONT OPTIONS --------------------
% sans-serif
% \usepackage[sfdefault]{roboto}
% \usepackage[sfdefault]{noto-sans}
% serif
% \usepackage{charter}

\pagestyle{fancy}
\fancyhf{} % clear all header and footer fields
\fancyfoot{}
\renewcommand{\headrulewidth}{0pt}
\renewcommand{\footrulewidth}{0pt}

% Adjust margins
\addtolength{\oddsidemargin}{-0.5in}
\addtolength{\evensidemargin}{-0.5in}
\addtolength{\textwidth}{1in}
\addtolength{\topmargin}{-1in} % Default was -.5in
\addtolength{\textheight}{1.0in}

\urlstyle{same}

\raggedbottom
\raggedright
\setlength{\tabcolsep}{0in}

% Section formatting
\titleformat{\section}{
  \vspace{-5pt}\scshape\raggedright\large
}{}{0em}{}[\color{black}\titlerule \vspace{-5pt}]

% Subsection formatting
\titleformat{\subsection}{
  \vspace{-4pt}\scshape\raggedright\large
}{\hspace{-.15in}}{0em}{}[\color{black}\vspace{-8pt}]

% Ensure that generate pdf is machine readable/ATS parsable
\pdfgentounicode=1

% -------------------- CUSTOM COMMANDS --------------------
\newcommand{\resumeItem}[1]{
  \item\small{
    {#1 \vspace{-2pt}}
  }
}

\newcommand{\resumeSubheading}[4]{
  \vspace{-2pt}\item
    \begin{tabular*}{0.97\textwidth}[t]{l@{\extracolsep{\fill}}r}
      \textbf{#1} & #2 \\
      \textit{\small#3} & \textit{\small #4} \\
    \end{tabular*}\vspace{-7pt}
}

\newcommand{\resumeSubSubheading}[2]{
    \item
    \begin{tabular*}{0.97\textwidth}{l@{\extracolsep{\fill}}r}
      \textit{\small#1} & \textit{\small #2} \\
    \end{tabular*}\vspace{-7pt}
}

\newcommand{\resumeProjectHeading}[2]{
    \item
    \begin{tabular*}{0.97\textwidth}{l@{\extracolsep{\fill}}r}
      \small#1 & #2 \\
    \end{tabular*}\vspace{-7pt}
}

\newcommand{\resumeSubItem}[1]{\resumeItem{#1}\vspace{-4pt}}
\newcommand{\resumeSubHeadingListStart}{\begin{itemize}[leftmargin=0.15in, label={}]}
\newcommand{\resumeSubHeadingListEnd}{\end{itemize}}
\newcommand{\resumeItemListStart}{\begin{itemize}}
\newcommand{\resumeItemListEnd}{\end{itemize}\vspace{-5pt}}

\renewcommand\labelitemii{$\vcenter{\hbox{\tiny$\bullet$}}$}

\setlength{\footskip}{4.08003pt}

% -------------------- START OF DOCUMENT --------------------
\begin{document}

% -------------------- HEADING--------------------
\begin{flushright}
  % \vspace{-4pt}
  \color{gray}
  \item
  Last Updated on February 14th, 2024
\end{flushright}

\vspace{-5pt}

\begin{center}
    \textbf{\Huge \scshape Huzaifa Mustafa Unjhawala} \\ \vspace{8pt}
    \small 
    \faIcon{github}
    \href{https://github.com/Huzaifg}{\underline{https://github.com/Huzaifg}} $  $
    \faIcon{linkedin}
    \href{https://linkedin.com/in/unjhawala/}{\underline{linkedin.com/in/unjhawala/}} $  $
    \faIcon{envelope}
    \href{mailto:unjhawala@wisc.edu}
    {\underline{unjhawala@wisc.edu}}
\end{center}

% -------------------- EDUCATION --------------------
\section{Education}
  \resumeSubHeadingListStart
  
    \resumeSubheading
      {University of Wisconsin-Madison}{May 2026}
      {Ph.D. Mechanical Engineering, Minor in Mathematics}
      {Current GPA: 3.92/4.0}
    \resumeSubheading
      {University of Wisconsin-Madison}{Dec 2024}
      {MS Computer Science (Course-based)}
      {Current GPA: 3.9/4.0}  
    \resumeSubheading
      {University of Wisconsin-Madison}{May 2023}
      {MS Mechanical Engineering}
      {Current GPA: 3.94/4.0}
    \resumeSubheading
      {National Institute Of Technology - Trichy}{June 2020}
      {B.Tech with Honors in Mechanical Engineering}{GPA: 8.9/10.0}

    \vspace{-10pt}

    \subsection{Coursework}
      \textbf{Courses:} High Performance Computing, Scientific Computing, Non-Linear Finite Elements, Mechanics of Continua, Machine Learning, Stochastic Computational Methods, Non-linear Optimization, Kinematics and Dynamics of Machine Systems\\
      \textbf{Awards:} Baden-Württemberg Stipendium

  \resumeSubHeadingListEnd

% -------------------- SKILLS --------------------
\section{Skills}
 \begin{itemize}[leftmargin=0.15in, label={}]
    \small{\item{
    
     \textbf{Languages}{: C/C++, CUDA, Python (incl. JAX, PyTorch), Matlab, \LaTeX} \\
     
    \textbf{Tools}: Git, Linux (Arch, Ubuntu), docker, CMake, Shell (Bash, Zsh), SWIG
    }}
 \end{itemize}

% -------------------- PROJECTS --------------------
\section{Research Experience}
    \resumeSubHeadingListStart

        \resumeProjectHeading
        {\textbf{Low-Fidelity Vehicle Dynamic Models} $|$ \footnotesize\emph Simulation Based Engineering Lab, UW Madison}{Jan 2022 -- May 2023}
        \resumeItemListStart
            \resumeItem{Developed a library of Low-Fidelity Vehicle Models that are 1000x faster than real-time on a CPU}
            \resumeItem{Parallelized the code using CUDA, achieving simulation of 300,000 vehicles in real-time}
            \resumeItem{Used a SWIG wrapper to provide a Python API to the model}
            \resumeItem{Used Bayesian Optimization to tune the parameters of the model to match real-world data and data from high-fidelity vehicle models}
            \resumeItem{Open source code can be found \href{https://github.com/uwsbel/low-fidelity-dynamic-models}{here}}
          \resumeItemListEnd

          \resumeProjectHeading
          {\textbf{Fast Terramechanics Simulation} $|$ \footnotesize\emph{Simulation Based Engineering Lab, UW Madison}}{Feb 2024 -- Present}
          \resumeItemListStart
              \resumeItem{Exploring the use of Graph-Neural Network and Transformer based models for fast terramechanics simulations with the main goal to enable real-time simulation of large-scale terrains for designing the autonomy stacks of construction equipment}
              \resumeItem{Early results can be found \href{https://uwmadison.box.com/s/o7j9ab15zl0u8vsjfe8958kot30c2khz}{here}}
          \resumeItemListEnd
    
        \resumeProjectHeading
        {\textbf{GymChrono} $|$ \footnotesize\emph{Simulation Based Engineering Lab, UW Madison}}{May 2023 -- Present}
        \resumeItemListStart
            \resumeItem{Co-maintaining the open-source Gymnasium environment for Project Chrono, a physics-based simulation engine for use in Reinforcement Learning applications}
            \resumeItem{Co-hosted a training session at \href{https://sbel.wisc.edu/magic-2023/}{MaGIC} whose slides can be found \href{https://uwmadison.box.com/s/l4ahegpgm01lo5wl2tgai05futrtyda8}{here}}
            \resumeItem{Open source code can be found \href{https://github.com/projectchrono/gym-chrono}{here}}

          \resumeItemListEnd
          
      \resumeProjectHeading
        {\textbf{Sensor Simulation Validation} $|$ \footnotesize\emph{Simulation Based Engineering Lab, UW Madison}}{Dec 2023 -- Mar 2024}
        \resumeItemListStart
            \resumeItem{Validating GPS and IMU sensor's in simulators such as AirSim and Project Chrono for velocity estimation using a novel contextual performance difference based approach}
            %\resumeItem{The sensor's are validated through a "judge", which in our case is EKF based velocity estimator}
            %\resumeItem{The velocity estimated by the EKF is compared to the ground truth velocity, in both simulation and real-world scenarios. These errors are then compared statistically to determine the performance of the sensors in simulation.}
        \resumeItemListEnd

        \resumeProjectHeading
        {\textbf{Calibration of Terramechanics Models}}{Jan 2023 -- May 2023}
        \resumeItemListStart
            \resumeItem{Contributed to the Bayesian Calibration of the Soil Contact Model (SCM) with the use of data generated with a virtual bevameter test for high-fidelity terramechanics simulations}
            %\resumeItem{The resulting calibrated model was fast and accurate enough to be used in real-time simulations of full rover simulations on granular terrains}

        \resumeItemListEnd

        \resumeProjectHeading
        {\textbf{Undergraduate Experience in Simulation}}{May 2018 -- May 2020}
        \resumeItemListStart
            \resumeItem{Awarded the Baden-Württemberg Stipendium for a 3-month research internship at the Karlsruhe Institute of Technology, Germany, where I worked on multi-body simulation models for axial thrust bearings in MSC Adams}
            \resumeItem{Built a transient fluid flow simulation model that was used to optimize the pressure drop of a Magneto-Rheological Damper in ANSYS CFX and AIM}
            \resumeItem{As part of the SAE Baja team, I was responsible for the design and simulation of the wheel assembly, achieving 20\% weight reduction while maintaining the same strength}
        \resumeItemListEnd

          
    \resumeSubHeadingListEnd

% -------------------- EXPERIENCE --------------------
\section{Relevant Work Experience}
  \resumeSubHeadingListStart

          \resumeProjectHeading
          {\textbf{National Renewable Energy Laboratory} $|$ \footnotesize\emph{Graduate Engineering Intern}\vspace{8pt}}{Jul. 2023 -- Sep. 2023}
        \resumeItemListStart
          \resumeItem{Contributed to HydroChrono - a C++ library for enabling Wave Energy Converter (WEC) simulations with Project Chrono}
          \resumeItem{Refactored code and setup testing infrastructure for the library}
          \resumeItem{Explored the use of multi-fidelity models for WEC simulations by enabling seamless transition from potential flow solvers used in HydroChrono to high-fidelity SPH solvers used in Project Chrono}
          \resumeItem{Open source code can be found \href{https://github.com/NREL/HydroChrono}{here}}
        \resumeItemListEnd
    \resumeSubHeadingListEnd 

% -------------------- PUBLICATIONS --------------------
\section{Publications}
  \resumeSubHeadingListStart

          \resumeProjectHeading
          {\textbf{Journal Publications}}{}
        \resumeItemListStart
            \resumeItem{Hu, W., Li, P., \textbf{Unjhawala, H.M.}, Serban, R. \& Negrut, D. (2023) \textbf{Calibration of an expeditious terramechanics model using a higher-fidelity model, Bayesian inference, and a virtual bevameter test.} Journal of Field Robotics, 1–20. \url{https://doi.org/10.1002/rob.22276}}
            \resumeItem{\textbf{Unjhawala, H. M.}, Zhang, R., Hu, W., Wu, J., Serban, R., and Negrut, D. (April 8, 2023). \textbf{Using a Bayesian-Inference Approach to Calibrating Models for Simulation in Robotics.} ASME. J. Comput. Nonlinear Dynam. June 2023; 18(6): 061004. \url{https://doi.org/10.1115/1.4062199}}
            \resumeItem{\textbf{H. Unjhawala} et al., \textbf{An Expeditious and Expressive Vehicle Dynamics Model for Applications in Controls and Reinforcement Learning}, in IEEE Access, vol. 12, pp. 33000-33015, 2024, doi: 10.1109/ACCESS.2024.3368874.\url{https://ieeexplore.ieee.org/document/10443432} }
        \resumeItemListEnd
        \resumeProjectHeading{\textbf{Conference Publications}}{}
        \resumeItemListStart
        \resumeItem{Zhou, Z., \textbf{Unjhawala, H.}, Kamaraj, A., Kissel, A., Lee, J., Serban, R., Negrut, D., ''A Chrono-Based Framework for Large-Scale Traffic Simulation with Human-In-The-Loop.'' Proceedings of the Multibody 2023 11th ECCOMAS Thematic Conference on Multibody Dynamics, Lisboa, Portugal. July 24-28, 2023. Preprint \url{https://doi.org/10.13140/RG.2.2.23133.59361}}
        \resumeItemListEnd

        \resumeProjectHeading{\textbf{Under Review}}{}
        \resumeItemListStart
        \resumeItem{\textbf{Unjhawala, H.}, Mahajan, I., Serban, R., Negrut, D., \textbf{Fast and Accurate Low Fidelity Dynamic Models for Robotics}, Journal of Open Source Software, Preprint \url{https://github.com/uwsbel/low-fidelity-dynamic-models/blob/12-complete-paper-for-joss/paper/paper.md}}
        \resumeItem{Ishaan Mahajan, \textbf{Huzaifa Unjhawala}, Harry Zhang, Zhenhao Zhou, Aaron Young, Alexis Ruiz, Stefan Caldararu, Nevinu Batagoda, Sriram Ashokkumar, and Dan Negrut. \textbf{Quantifying the Sim2Real Gap for GPS and IMU Sensors.}, Under Review IROS 2024}
        \resumeItem{Zhang, H., Caldararu, Young, A., Ruiz, A., \textbf{Unjhawala, H.},  Mahajan, I., S., Ashokkumar, S., Bakke, L., Negrut, D., \textbf{A Study on the Use of Simulation in Synthesizing Path-Following Control Policies for Autonomous Ground Robots}, Under Review IROS 2024}
        \resumeItemListEnd
    \resumeSubHeadingListEnd

\end{document}